\documentclass[journal]{IEEEtran}
\usepackage[a5paper, margin=10mm, onecolumn]{geometry}
\usepackage{tfrupee} % Include tfrupee package
\usepackage{cite}
\usepackage{amsmath,amssymb,amsfonts,amsthm}
\usepackage{algorithmic}
\usepackage{graphicx}
\usepackage{textcomp}
\usepackage{xcolor}
\usepackage{txfonts}
\usepackage{enumitem}
\usepackage{mathtools}
\usepackage{gensymb}
\usepackage{comment}
\usepackage[breaklinks=true]{hyperref}
\usepackage{tkz-euclide} 
\usepackage[latin1]{inputenc}                                
\usepackage{color}                                            
\usepackage{array}                                            
\usepackage{longtable}                                       
\usepackage{calc}                                             
\usepackage{multirow}                                         
\usepackage{hhline}                                           
\usepackage{ifthen}                                           
\usepackage{lscape}

\setlength{\headheight}{1cm} % Set the height of the header box
\setlength{\headsep}{0mm}     % Set the distance between the header box and the top of the text

\begin{document}

\bibliographystyle{IEEEtran}
\renewcommand{\thefigure}{\theenumi}
\renewcommand{\thetable}{\theenumi}
\setlength{\intextsep}{10pt} % Space between text and floats
\numberwithin{equation}{enumi}
\numberwithin{figure}{enumi}
\title{4-4.2-21}
\author{AI24BTECH11016-Jakkula Adishesh Balaji}
\maketitle

\section*{\textbf{Linear Forms (Parameters)}}
\parindent 0pt
Question: \\
\textbf{4.2.21} Find the direction and normal vectors of the line \( F = \frac{9}{5}C + 32 \).

\begin{table}[h!]    	
    \centering
    \begin{tabular}{|c|c|}
    \hline
    P. Maximum-normal-stress criterion & \input{figs/figs.tex} \\ 
    \hline
    Q. Maximum-distortion-energy criterion & \input{figs/figs1.tex} \\
    \hline
    R. Maximum-shear-stress criterion & 
\begin{circuitikz}[scale = 0.5]
\tikzstyle{every node}=[font=\normalsize]
\draw [short] (5.25,12.25) .. controls (9.25,13.5) and (10.75,6) .. (5.25,6.75);
\draw [short] (8.25,9.25) -- (9.25,9.25);
\draw [short] (4.5,9.25) -- (6.5,9.25);
\draw [short] (4.5,9.25) -- (4.25,8.75);
\draw [short] (5,9.25) -- (4.75,8.75);
\draw [short] (5.5,9.25) -- (5.25,8.75);
\draw [short] (6,9.25) -- (5.75,8.75);
\draw [short] (6.5,9.25) -- (6.25,8.75);
\draw [->, >=Stealth] (8.75,9.25) -- (6.75,11.5);
\draw [->, >=Stealth] (8.75,13.5) -- (8.75,11.5);
\node [font=\normalsize] at (4.5,9.5) {P};
\node [font=\normalsize] at (8.75,14) {Q};
\node [font=\normalsize] at (8.75,11.25) {W};
\node [font=\normalsize] at (7.5,10) {R};
\end{circuitikz}

 \\
    \hline
\end{tabular}


    \caption{Variables Used}
    \label{tab1-1.9-6}
\end{table}

\solution
The normal vector can be found as follows:
\begin{align}
    F - \frac{9}{5}C &= 32 \\
    \begin{pmatrix} 1 & -\frac{9}{5} \end{pmatrix} \begin{pmatrix} F \\ C \end{pmatrix} &= \begin{pmatrix} 32 \\ 0 \end{pmatrix} \\
    \begin{pmatrix} 1 & -\frac{9}{5} \end{pmatrix} x &= 32 \\
    \vec{n}^\top \vec{x} &= 32\\
    \vec{n} &= \begin{pmatrix} 1 \\ -\frac{9}{5} \end{pmatrix}
\end{align}

The direction vector can be found as follows:
\begin{align}
    F &= F \\
    C &= \frac{5}{9}F - \frac{160}{9} \\
    \begin{pmatrix} F \\ C \end{pmatrix} &= F \begin{pmatrix} 1 \\ \frac{5}{9} \end{pmatrix} + \begin{pmatrix} 0 \\ -\frac{160}{9} \end{pmatrix} \\
    \vec{x} &= k \vec{m} + \vec{h} \\
    \vec{m} &= \begin{pmatrix} 1 \\ \frac{5}{9} \end{pmatrix}
\end{align}

Direction vector: \( \vec{m} = \begin{pmatrix} 1 \\ \frac{5}{9} \end{pmatrix} \\
\text{Normal vector: } \vec{n} = \begin{pmatrix} 1 \\ -\frac{9}{5} \end{pmatrix} \)

\begin{figure}[h!]
    \centering
    \includegraphics[width = 1\linewidth]{figs/fig.png}
    \caption{Graphical representation of the vectors.}
    \label{stemplot}
\end{figure}

\end{document}

