%iffalse
\documentclass[journal]{IEEEtran}
\usepackage[a5paper, margin=10mm, onecolumn]{geometry}
%\usepackage{lmodern} % Ensure lmodern is loaded for pdflatex
\usepackage{tfrupee} % Include tfrupee package

\setlength{\headheight}{1cm} % Set the height of the header box
\setlength{\headsep}{0mm}     % Set the distance between the header box and the top of the text

\usepackage{gvv-book}
\usepackage{gvv}
\usepackage{cite}
\usepackage{amsmath,amssymb,amsfonts,amsthm}
\usepackage{algorithmic}
\usepackage{graphicx}
\usepackage{textcomp}
\usepackage{xcolor}
\usepackage{txfonts}
\usepackage{listings}
\usepackage{enumitem}
\usepackage{mathtools}
\usepackage{gensymb}
\usepackage{comment}
\usepackage[breaklinks=true]{hyperref}
\usepackage{tkz-euclide} 
\usepackage{listings}
% \usepackage{gvv}                                        
\def\inputGnumericTable{}                                 
\usepackage[latin1]{inputenc}                                
\usepackage{color}                                            
\usepackage{array}                                            
\usepackage{longtable}                                       
\usepackage{calc}                                             
\usepackage{multirow}                                         
\usepackage{hhline}                                           
\usepackage{ifthen}                                           
\usepackage{lscape}
\begin{document}
\bibliographystyle{IEEEtran}
\title{2021-August Session-31-08-2021 shift 2}
\author{AI24BTECH11016-Jakkula Adishesh Balaji}
{\let\newpage\relax\maketitle}
\renewcommand{\thefigure}{\theenumi}
\renewcommand{\thetable}{\theenumi}
\setlength{\intextsep}{10pt} % Space between text and floats
\numberwithin{equation}{enumi}
\numberwithin{figure}{enumi}
\renewcommand{\thetable}{\theenumi}
\section{16-30(Math)}
\begin{enumerate}
	\item
	Let $A$ be the set of all points $\myvec{\alpha , \beta}$ such that the area of triangle formed by the points $\myvec{5 , 6}$ $\myvec{3 , 2}$ and $\myvec{\alpha , \beta}$ is 12 square units. Then the least possible length of a line segment joining the origin to a point in $A$, is 
		\begin{enumerate}
			\item  $\frac{4}{\sqrt{5}}$
			\item  $\frac{16}{\sqrt{5}}$
			\item  $\frac{8}{\sqrt{5}}$
			\item  $\frac{12}{\sqrt{12}}$
		\end{enumerate}
	\item
	The number of solutions of the equation $32^{\tan^{2}{x}} + 32^{\sec^{2}{x}} = 81$, $0 \leq x \leq \frac{\pi}{4}$ is:
		\begin{enumerate}
			\item 3
			\item 1
			\item 0
			\item 2
		\end{enumerate}
	\item 
	Let $f$ be any continuous function on [0, 2] and twice differentiable on (0, 2). If $f\brak{0} = 0$, $f\brak{1} = 1$ and $f\brak{2} = 2$, then
		\begin{enumerate}
			\item  $f^{"}\brak{x} = 0$ for all $x \in \brak{0, 2}$
			\item  $f^{"}\brak{x} = 0$ for some $x \in \brak{0, 2}$
			\item  $f^{'}\brak{x} = 0$ for some $x \in \sbrak{0, 2}$
			\item  $f^{"}\brak{x} > 0$ for all $x \in \brak{0, 2}$
		\end{enumerate}
	\item 
	If $\text{step}(x)$ is the greatest integer $\leq x$, then $\pi^{2}\int_{0}^{2} \sin{\frac{\pi x}{2}}(x-\sbrak{x})^{\sbrak{x}} dx$ is equal to;
		\begin{enumerate}
			\item $2\brak{\pi-1}$
			\item $4\brak{\pi-1}$
			\item $4\brak{\pi+1}$
			\item $2\brak{\pi+1}$
		\end{enumerate}
	\item 
	The mean and variance of 7 observations are 8 and 16 respectively. If two observations are 6 and 8, then the variance of the remaining 5 observations is:
		\begin{enumerate}
			\item $\frac{92}{5}$
			\item $\frac{134}{5}$
			\item $\frac{536}{25}$
			\item $\frac{112}{5}$
		\end{enumerate}
    \item 
    If the coefficient of $a^{7}b^{8}$ in the expansion of $(a + 2b + 4ab)^{10}$ is $K.2^{16}$, then $K$ is equal to
    \item
	Suppose the line $\frac{x-2}{\alpha} = \frac{y-2}{-5} = \frac{z+2}{2}$ lies on the plane  $x + 3y - 2z + \beta = 0$. Then the value of $\alpha + \beta$ is equal to 
	\item 
	The number of 4-digit numbers which are neither multiple of 7 nor multiple of 3 is 
	\item 
	$\int \frac{\sin{x}}{\sin^3{x} + \cos^3{x}} dx =$ when $C$ is a constant of integration, then the value of $18\brak{\alpha+\beta+\gamma^{2}}$ is
	\item
	A tangent line L is drawn at the point $\myvec{2 , 4}$ on the parabola $y^{2}= 8x$. If the line L is also tangent to the circle $x^2 + y^2 = a$, then 'a' is equal to
	\item 
	If $S=\frac{7}{5} + \frac{9}{5^{2}} + \frac{13}{5^{3}} + \frac{19}{5^{4}} + ....$ then $160S$ is equal to 
	\item 
	The number of elements in the set  \\
	$
	\left\{
	A = \myvec{a & b \\ 0 & d}: a,b,d \in \{-1,0,1\} and \brak{I-A}^{3} = I - A^{3}
	\right\}
	$
	\item 
	If the line $y = mx$ bisects the area enclosed by the lines $x = 0$, $y = 0$, $x = \frac{3}{2}$ and the curve $y = 1 + 4x - x^{2}$, then $12m$ is equal to 
	\item 
	Let $B$ be the centre of the circle $x^2+ y^2 - 2x + 4y + 1 = 0$. Let the tangents at two points $\vec{P}$ and $\vec{Q}$ on the circle intersect at the point $\vec{A}\myvec{3 , 1}$. Then $\frac{Area of \Delta APQ}{Area of \Delta BPQ}$ is equal to
	\item 
 	Let $f\brak{x}$ be a cubic polynomial with $f\brak{1} = -10$, $f\brak{-1} = 6$, and has a local minima at $x = 1$, and $f^{'}\brak{x}$ has a local minima at $x = -1$. Then $f\brak{3}$ is equal to 
\end{enumerate}
\end{document}
