\documentclass[journal]{IEEEtran}
\usepackage[a5paper, margin=10mm, onecolumn]{geometry}
\usepackage{tfrupee} % Include tfrupee package

\setlength{\headheight}{1cm} % Set the height of the header box
\setlength{\headsep}{0mm}     % Set the distance between the header box and the top of the text

\usepackage{gvv-book}
\usepackage{gvv}
\usepackage{cite}
\usepackage{amsmath,amssymb,amsfonts,amsthm}
\usepackage{algorithmic}
\usepackage{graphicx}
\usepackage{textcomp}
\usepackage{xcolor}
\usepackage{txfonts}
\usepackage{listings}
\usepackage{enumitem}
\usepackage{mathtools}
\usepackage{gensymb}
\usepackage{comment}
\usepackage[breaklinks=true]{hyperref}
\usepackage{tkz-euclide}
\usepackage{longtable}
\usepackage{lscape}

\def\inputGnumericTable{}                                 
\usepackage[latin1]{inputenc}                                
\usepackage{color}                                            
\usepackage{array}                                            
\usepackage{calc}                                             
\usepackage{multirow}                                         
\usepackage{hhline}                                           
\usepackage{ifthen}

\begin{document}
\bibliographystyle{IEEEtran}
\title{2023-April Session-04-11-2023 Shift 1}
\author{AI24BTECH11016 - Jakkula Adishesh Balaji}
{\let\newpage\relax\maketitle}

\renewcommand{\thefigure}{\theenumi}
\renewcommand{\thetable}{\theenumi}
\setlength{\intextsep}{10pt} % Space between text and floats
\numberwithin{equation}{enumi}
\numberwithin{figure}{enumi}

\section{16-30 (Math)}
\begin{enumerate}
	\item
	If the equation of the plane that contains the point $\brak{-2, 3, 5}$ and is perpendicular to each of the planes $2x + 4y + 5z = 8$ and $3x - 2y + 3z = 5$ is $\alpha x + \beta y + \gamma z + 97 = 0$, then $\alpha + \beta + \gamma$ is:
		\begin{enumerate}
			\item 15
			\item 18
			\item 17
			\item 16
		\end{enumerate}

	\item
	An organization awarded 48 medals in event '$A$', 25 in event '$B$' and 18 in event '$C$'. If these medals went to a total of 60 men and only five men got medals in all the three events, then how many received medals in exactly two of three events?
		\begin{enumerate}
			\item 15
			\item 9
			\item 21
			\item 10
		\end{enumerate}

	\item
	Let $y = y\brak{x}$ be a solution curve of the differential equation $\brak{1 - x^{2}y^{2}}dx = y dx + x dy$. If the line $x = 1$ intersects the curve $y = y\brak{x}$ at $y = 2$ and the line $x = 2$ intersects the curve $y = y\brak{x}$ at $y = \alpha$, then the value of $\alpha$ is:
		\begin{enumerate}
			\item $\frac{1+3e^{2}}{2\brak{3e^{2}-1}}$
			\item $\frac{1-3e^{2}}{2\brak{3e^{2}+1}}$
			\item $\frac{3e^{2}}{2\brak{3e^{2}-1}}$
			\item $\frac{3e^{2}}{2\brak{3e^{2}+1}}$
		\end{enumerate}

	\item 
	Let $\brak{\alpha, \beta, \gamma}$ be the image of the point $\vec{P} = \brak{2, 3, 5}$ in the plane $2x + y - 3z = 6$. Then $\alpha + \beta + \gamma$ is equal to:
		\begin{enumerate}
			\item 5
			\item 9
			\item 10
			\item 12
		\end{enumerate}

	\item
	Let $f(x) = \sbrak{x^2 - x} + \abs{-x + \sbrak{x}}$, where $x \in \mathbb{R}$ and $\sbrak{t}$ denotes the greatest integer less than or equal to $t$. Then $f$ is:
		\begin{enumerate}
			\item not continuous at $x = 0$ and $x = 1$
			\item continuous at $x = 0$ and $x = 1$
			\item continuous at $x = 1$, but not continuous at $x = 0$
			\item continuous at $x = 0$, but not continuous at $x = 1$
		\end{enumerate}

	\item 
	The number of integral terms in the expansion of $\brak{3^{\frac{1}{2}} + 5^{\frac{1}{4}}}$ is:

	\item 
	The number of ordered triplets of the truth values of $p$, $q$, and $r$ such that the truth value of the statement $\brak{p \lor q} \land \brak{p \lor r} \implies \brak{q \lor r}$ is true, is equal to:

	\item
	Let $A = \myvec{0 & 1 & 2 \\ a & 0 & 3 \\ 1 & c & 0}$, where $a, c \in \mathbb{R}$. If $A^{3} = A$ and the positive value of $a$ belongs to the interval $(n-1,n]$ where $n \in \mathbb{N}$, then $n$ is equal to:

	\item 
	For $m, n > 0$, let $\alpha\brak{m, n} = \int_{0}^{2} t^{m}\brak{1+3t}^{n} dt$. If $11\alpha\brak{10, 6} + 18\alpha\brak{11, 5} = p 14^{6}$ , then $p$ is equal to:

	\item 
	Let $S = 109 + \frac{108}{5} + \frac{107}{5^2} + \dots + \frac{2}{5^{107}} + \frac{1}{5^{108}}$. Then the value of $16S - \brak{25^{-54}}$ is equal to:

	\item 
	Let $H_n:\frac{x^2}{1+n} - \frac{y^2}{3+n} = 1$, $n \in \mathbb{N}$. Let $k$ be the smallest even value of $n$ such that the eccentricity of $H_k$ is a rational number. If $l$ is the length of the latus rectum of $H_k$, then $21l$ is equal to:

	\item
	The mean of the coefficients of $x$, $x^2$,..., $x^7$ in the binomial expansion of $\brak{2 + x}^{9}$ is:

	\item
	If $a$ and $b$ are the roots of the equation $x^{2} - 7x - 1 = 0$, then the value of $\frac{a^{21} + b^{21} + a^{17} + b^{17}}{a^{19} + b^{19}}$ is equal to:

	\item
	In an examination, 5 students have been allotted their seats as per their roll numbers. The number of ways in which none of the students sits on the allotted seat is:

	\item
Let a line $l$ pass through the origin and be perpendicular to the lines \\
	$l_1 : \vec{r} = \hat{i} - 11\hat{j} - 7\hat{k} + \lambda\brak{\hat{i} +2\hat{j} + 3\hat{k}}, \lambda \in \mathbb{R}$ and \\
	$l_2 : \vec{r} = -\hat{i} + \hat{k} + \mu\brak{2\hat{i} +2\hat{j} + \hat{k}}, \mu \in \mathbb{R}$
	If $\vec{P}$ is the point of intersection of $l$ and $l_1$, and $\vec{Q} \myvec{\alpha & \beta & \gamma}$ is the foot of perpendicular from $\vec{P}$ on $l_2$, then $9\brak{\alpha+\beta+\gamma}$ is equal to
 \end{enumerate}
 \end{document}
