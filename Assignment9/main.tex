\documentclass[journal]{IEEEtran}
\usepackage[a5paper, margin=10mm, onecolumn]{geometry}
\usepackage{tfrupee} % Include tfrupee package

\setlength{\headheight}{1cm} % Set the height of the header box
\setlength{\headsep}{0mm}     % Set the distance between the header box and the top of the text

\usepackage{gvv-book}
\usepackage{gvv}
\usepackage{cite}
\usepackage{amsmath,amssymb,amsfonts,amsthm}
\usepackage{algorithmic}
\usepackage{graphicx}
\usepackage{textcomp}
\usepackage{xcolor}
\usepackage{txfonts}
\usepackage{listings}
\usepackage{enumitem}
\usepackage{mathtools}
\usepackage{gensymb}
\usepackage{comment}
\usepackage[breaklinks=true]{hyperref}
\usepackage{tkz-euclide}
\usepackage{longtable}
\usepackage{lscape}

\def\inputGnumericTable{}                                 
\usepackage[latin1]{inputenc}                                
\usepackage{color}                                            
\usepackage{array}                                            
\usepackage{calc}                                             
\usepackage{multirow}                                         
\usepackage{hhline}                                           
\usepackage{ifthen}

\begin{document}
\bibliographystyle{IEEEtran}
\title{2007-GATE-CE}
\author{AI24BTECH11016 - Jakkula Adishesh Balaji}
{\let\newpage\relax\maketitle}

\renewcommand{\thefigure}{\theenumi}
\renewcommand{\thetable}{\theenumi}
\setlength{\intextsep}{10pt} % Space between text and floats
\numberwithin{equation}{enumi}
\numberwithin{figure}{enumi}
\section{1-17}
\begin{enumerate}
	\item
	The minimum and the maximum eigen values of the matrix $\myvec{1 & 1 & 3 \\ 1 & 5 & 1 \\ 3 & 1 & 1}$ are -2 and 6, respectively. What is the other eigen value?
		\begin{enumerate}
			\item 5
			\item 3
			\item 1
			\item -1
		\end{enumerate}
	\item
	The degree of the differential equation $\frac{d^{2}x}{d{t}^2} + 2x^3 = 0$ is
		\begin{enumerate}
			\item 0
			\item 1
			\item 2
			\item 3
		\end{enumerate}
	\item
	The solution for the differential equation $\frac{dy}{dx} = x^{2}y$ with the condition that $y=1$ at $x=0$ is 
		\begin{enumerate}
			\item $y=e^{\frac{1}{2x}}$
			\item $\ln\brak{y} = \frac{x^3}{3} + 4$
			\item $\ln\brak{y} = \frac{x^2}{2}$
			\item $y = e^{\frac{x^3}{3}}$
		\end{enumerate}
	\item
	An axially loaded bar is subjected to a normal stress of $173 MPa$. The shear stress in the bar is
		\begin{enumerate}
			\item $75 MPa$
			\item $86.5 MPa$
			\item $100 MPa$
			\item $122.3 MPa$
		\end{enumerate}
	\item
	A steel column, pinned at both ends, has a buckling load of $200 kN$. If the column is restrained against lateral movement at its mid-height, its buckling load will be 
		\begin{enumerate}
			\item $200 kN$
			\item $283 kN$
			\item $400 kN$
			\item $800 kN$
		\end{enumerate}
	\item
	The stiffness coefficient $k_{ij}$ indicates
		\begin{enumerate}
			\item force at $i$ due to a unit deformation at $j$
			\item deformation at $j$ due to a unit force at $i$
			\item deformation at $i$ due to a unit force at $j$
			\item force at $j$ due to a unit deformation at $i$
		\end{enumerate}
	\item
	For an isotropic material, the relationship between the Young's modulus $\brak{E}$, shear modulus $\brak{\mu}$ is given by 
		\begin{enumerate}
			\item $G = \frac{E}{2\brak{1+\mu}}$
			\item $E = \frac{G}{2\brak{1+\mu}}$
			\item $G = \frac{E}{1 + 2\mu}$
			\item $G = \frac{E}{2\brak{1-\mu}}$
		\end{enumerate}
	\item
	A clay soil sample is tested in a triaxial apparatus in consolidated-drained conditions at a cell pressure of $100 kN/m^2$. What will be the pore water pressure at a deviator stress of $40 kN/m^2$
		\begin{enumerate}
			\item 0 $kN/m^2$
			\item 20 $kN/m^2$
			\item 40 $kN/m^2$
			\item 60 $kN/m^2$
		\end{enumerate}
	\item
	The number of blows observed in a Standard Penetration Test (STP) for different penetration depths are given as follows: \\
	\begin{table}[h!]    	
    		\centering
    % Assuming the table.tex file exists
    		\begin{tabular}[12pt]{ |c| c|}
    \hline
    Parameter & Description\\ 
    \hline
    $P$ & $\myvec{2\\-3}$ \\
    \hline 
    $Q$ & $\myvec{10\\y}$ \\
    \hline
    $D$ & $Q-P$\\
    \hline 
    Distance & $10$ \\
    \hline
    \end{tabular}
 
       \end{table}
	\\
	The observed $N$ value is
		\begin{enumerate}
			\item 8
			\item 14
			\item 18
			\item 24
		\end{enumerate}
	\item
	The vertical stress at some depth below the corner of a $2m \times 3m$ rectangular footing due to a certain load intensity is $100 kN/m^2$. What will be the vertical stress in $kN/m^2$ below the centre of a $4m \times 6m$ rectangular footing at the same depth and same load intensity?
		\begin{enumerate}
			\item 25
			\item 100
			\item 200
			\item 400
		\end{enumerate}
	\item
	There is a free overfall at the end of a long open channel. For a given flow rate, the critical depth is less than the normal depth. What gradually varied flow profile will occur in the channel for this flow rate?
		\begin{enumerate}
			\item $M_1$
			\item $M_2$
			\item $M_3$
			\item $S_1$
		\end{enumerate}
	\item
	The consecutive use of water for a crop during a particular stage of growth is $2.0 mm/day$. The maximum depth of available water in the root zone is 60 mm. Irrigation is required when the amount of available water is 50 \% of the maximum available water in the root zone. Frequency of irrigation should be
		\begin{enumerate}
			\item 10 days
			\item 15 days
			\item 20 days
			\item 25 days
		\end{enumerate}
	\item
	As per the Lacey's method for design of alluvial channels, identify the \textbf{TRUE} statement from the following:
		\begin{enumerate}
			\item Wetted perimeter increases with an increase in design discharge
			\item Hydraulic radius increases with an increase in silt factor
			\item Wetted perimeter decreases with an increase in design discharge
			\item Wetted perimeter increases with an increase in silt factor
		\end{enumerate}
	\item
	At two points 1 and 2 in a pipeline the velocities are $V$ and $2V$, respectively. Both the points are at the same elevation. The fluid density is $\rho$. The flow can be assumed to be incompressible, inviscid, steady and irrotational. The difference is pressures $P_1$ and $P_2$ at points 1 and 2 is
		\begin{enumerate}
			\item $0.5\rho V^2$
			\item $1.5\rho V^2$
			\item $2\rho V^2$
			\item $3\rho V^2$
		\end{enumerate}
	\item
	The presence of hardness in excess of permissible limit causes 
		\begin{enumerate}
			\item cardio vascular problems
			\item skin discolouration
			\item calcium deficiency
			\item increased laundry expenses
		\end{enumerate}
	\item
	The dispersion of pollutants in atmosphere is maximum when
		\begin{enumerate}
			\item environmental lapse rate is greater than adiabatic lapse rate.
			\item environmental lapse rate is less than adiabatic lapse rate.
			\item environmental lapse rate is equal to adiabatic lapse rate.
			\item maximum mixing depth is equal to zero.
		\end{enumerate}
	\item 
	The alkalinity and the hardness of a water sample are $250 mg/L$ and $350 mg/L$ as $\text{CaCO}_3$, respectively. The water has 
		\begin{enumerate}
			\item $350 mg/L$ carbonate hardness and zero non-carbonate hardness.
			\item $250 mg/L$ carbonate hardness and zero non-carbonate hardness.
			\item $250 mg/L$ carbonate hardness and $350 mg/L$ non-carbonate hardness.
			\item $350 mg/L$ carbonate hardness and $100 mg/L$ non-carbonate hardness.
		\end{enumerate}
	
	







\end{enumerate}
\end{document}
