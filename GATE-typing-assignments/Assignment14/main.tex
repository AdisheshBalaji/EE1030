\documentclass[journal]{IEEEtran}
\usepackage[a5paper, margin=10mm, onecolumn]{geometry}
\usepackage{tfrupee} % Include tfrupee package

\setlength{\headheight}{1cm} % Set the height of the header box
\setlength{\headsep}{0mm}     % Set the distance between the header box and the top of the text

\usepackage{gvv-book}
\usepackage{gvv}
\usepackage{cite}
\usepackage{amsmath,amssymb,amsfonts,amsthm}
\usepackage{algorithmic}
\usepackage{graphicx}
\usepackage{textcomp}
\usepackage{xcolor}
\usepackage{txfonts}
\usepackage{listings}
\usepackage{enumitem}
\usepackage{mathtools}
\usepackage{gensymb}
\usepackage{comment}
\usepackage[breaklinks=true]{hyperref}
\usepackage{tkz-euclide}
\usepackage{longtable}
\usepackage{lscape}

\def\inputGnumericTable{}                                 
\usepackage[latin1]{inputenc}                                
\usepackage{color}                                            
\usepackage{array}                                            
\usepackage{calc}                                             
\usepackage{multirow}                                         
\usepackage{hhline}                                           
\usepackage{ifthen}
\usepackage{tikz}

\begin{document}
\bibliographystyle{IEEEtran}
\title{2023-GATE-CE}
\author{AI24BTECH11016 - Jakkula Adishesh Balaji}
{\let\newpage\relax\maketitle}

\renewcommand{\thefigure}{\theenumi}
\renewcommand{\thetable}{\theenumi}
\setlength{\intextsep}{10pt} % Space between text and floats
\numberwithin{equation}{enumi}
\numberwithin{figure}{enumi}
\section{53-65}
\begin{enumerate}
	\item
	The infitesimal element shown in the figure (not to scale) represents the state of stress at a point in a body. What is the magnitude of the maximum principal stress (in $N/mm^{2}$, in integer) at the point?
		%fig
	\item
	An idealised bridge truss is shown in the figure. The force in Member $U_{2}L_{3}$ is $kN$(round off to one decimal place).
		%fig
	\item 
	The cross-section of a girder is shown in the figure (not to scale). The section is symmetric about a vertical axis ($Y-Y$). The moment of inertia of the section about the horizontal axis ($X-X$) passing through the centroid is $cm^{4}$ (round off to nearest integer)
	%fig
	\item
	A soil having the average properties, bulk unit weight = $19 kN/m^{3}$; $angle of internal friction = 25 \degree $ and $cohesion = 15 kPa$, is being formed on a rock slope existing at an inclination of $35 \degree $ with the horizontal. The critical height (in m) of the soil formation up to which it would be stable without any failure is (round off to one decimal place).
	\\
	\sbrak{Assume the soil is being formed parallel to the rock bedding plane and there is no ground water effect.}
	\item
	A smooth vertical retaining wall supporting layered soils is shown in figure. According to Rankine's earth pressure theory, the lateral active earth pressure acting at the base of the wall is $kPa$ (round off to one decimal place.)
	%fig
	\item
	A vertical trench is excavated in a clayey soil deposit having a surcharge load of $30 kPa$. A fluid of unit weight $12 kN/m^{3}$ is poured in the trench to prevent collapse as the excavation proceeds. Assume that the fluid is not seeping through the soil deposit. If the undrained cohesion of the clay deposit is $20 kPa$ and saturated unit weight is $18 kN/m^{3}$, what is the maximum depth of unsupported excavation(in m, rounded off to two decimal places)?
	\item
	A 12-hour storm occurs over a catchment and results in a direct runoff depth of $100 mm$. The time-distribution of the rainfall intensity is shown in the figure(not to scale). The $\phi$-index of the storm is(in $mm$, rounded off to two decimal places)
	%fig
	\item
	A hydraulic jump occurs in a 1.0 m wide horizontal, frictionless, rectangular channel with a pre-jump depth of $0.2 m$ and a post-jump depth of $1.0 m$. The value of $g$ may be taken as $10 m/s^2$. The values of the specific force at the pre-jump and post-jump sections are same and are equal to (in $m^3$, rounded off to two decimal places)
	\item
	In Horton's equation fitted to the infiltration data for a soil, the initial infiltration capacity is $10 mm/h$; final infiltration capacity is $5 mm/h$; and the exponential decay constant is $0.5 /h$. Assuming that the infiltration takes place at capacity rates, the total infiltration depth (in $mm$) from a uniform storm of duration $12 h$ is (round off to one decimal place)
	\item
	The composition and energy content of a representative solid waste sample are given in the table. If the moisture content of the waste is $26 \%$, the energy content of the solid waste on dry-weight basis is $MJ/kg$(round off to one decimal place)
	%table
	\item
	A flocculator tank has a volume of $2800 m^3$. The temperature of water in the tank is $15 \degree C$, and the average velocity gradient maintained in the tank is $100/s$. The temperature of water is reduced to $5 \degree C$, but all other operating conditions including the power input are maintained as the same. The decrease in the average velocity gradient (in \%) due to the reduction in water temperature is (round off to the nearest integer)\\
	\sbrak{\text{Consider dynamic viscosity of water at}  15 \degree C \text{and}  5 \degree C  \text{as} 1.139 \times 10^{-3} N-s/m^{2} \text{and} 1.518 \times 10^{-3} N-s/m^{2}, \text{respectively}}
\end{enumerate}	
\end{document}
