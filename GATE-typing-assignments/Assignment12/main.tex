\documentclass[journal]{IEEEtran}
\usepackage[a5paper, margin=10mm, onecolumn]{geometry}
\usepackage{tfrupee} % Include tfrupee package

\setlength{\headheight}{1cm} % Set the height of the header box
\setlength{\headsep}{0mm}     % Set the distance between the header box and the top of the text

\usepackage{gvv-book}
\usepackage{gvv}
\usepackage{cite}
\usepackage{amsmath,amssymb,amsfonts,amsthm}
\usepackage{algorithmic}
\usepackage{graphicx}
\usepackage{textcomp}
\usepackage{xcolor}
\usepackage{txfonts}
\usepackage{listings}
\usepackage{enumitem}
\usepackage{mathtools}
\usepackage{gensymb}
\usepackage{comment}
\usepackage[breaklinks=true]{hyperref}
\usepackage{tkz-euclide}
\usepackage{longtable}
\usepackage{lscape}

\def\inputGnumericTable{}                                 
\usepackage[latin1]{inputenc}                                
\usepackage{color}                                            
\usepackage{array}                                            
\usepackage{calc}                                             
\usepackage{multirow}                                         
\usepackage{hhline}                                           
\usepackage{ifthen}
\usepackage{tikz}

\begin{document}
\bibliographystyle{IEEEtran}
\title{2013-GATE-CE}
\author{AI24BTECH11016 - Jakkula Adishesh Balaji}
{\let\newpage\relax\maketitle}

\renewcommand{\thefigure}{\theenumi}
\renewcommand{\thetable}{\theenumi}
\setlength{\intextsep}{10pt} % Space between text and floats
\numberwithin{equation}{enumi}
\numberwithin{figure}{enumi}
\section{27-39}
\begin{enumerate}
	\item
	Find the magnitude of the error (correct to two decimal places) in the estimation of following integral using Simpson's $\frac{1}{3}$ Rule. Take the step length as 1. \\
	$\int_{0}^{4} \brak{x^{4}+10} dx$
	\item
	The solution for $\int_{0}^{\frac{\pi}{6} \cos^{4}{3\theta}\sin^{3}{6\theta} d\theta$ is:
		\begin{enumerate}
			\item 0
			\item $\frac{1}{15}$
			\item 1
			\item $\frac{8}{3}$
		\end{enumerate}
	\item
	Find the value of $\lambda$ such that the function $f\brak{x}$ is a valid probability density function.
		\begin{align}
			f(x) =
			\begin{cases}
  				\lambda\brak{x-1}\brak{2-x} & \text{if } 1 \geq x \geq 2 \\
  				0 & \text{otherwise }
			\end{cases}
		\end{align}
	\item
	Laplace equation for water flow in soils is given below.
		\begin{align}
			\frac{\partial^2 H}{\partial x^2} + \frac{\partial^2 H}{\partial y^2} + \frac{\partial^2 H}{\partial z^2} &= 0
		\end{align}
	Head $H$ does not vary in $y$ and $z$ directions. \\
	Boundary conditions are: at $x = 0$, $H = 5$;and $\frac{dH}{dx} = -1$. \\
	What is the value of $H$ at $x = 1.2$?
	\item
	All members in the rigid-jointed frame shown are prismatic and have the same flexural stiffness $EI$
Find the magnitude of the bending moment at $Q$ (i $kNm$ ) due to the given loading

	%fig here
	\begin{figure}[H]
    		\centering
    		\input{figs/figs1.tex}
    		\caption{}
    		\label{36}
	\end{figure}
	
	\item
	A uniform beam ($EI = constant$)$PQ$ in the form of a quarter-circle of radius $R$ is fixed at end $P$ and free at the end $Q$, where, a load $W$ is applied as shown. The vertical downward displacement, $\delta_{q} = \beta\brak{\frac{W R^{3}}{EI}}$. Find the value of $\beta$(correct to 4-decimal places).
	
	%fig here
	\begin{figure}[H]
    		\centering
    		
\begin{circuitikz}[scale = 0.5]
\tikzstyle{every node}=[font=\normalsize]
\draw [short] (5.25,12.25) .. controls (9.25,13.5) and (10.75,6) .. (5.25,6.75);
\draw [short] (8.25,9.25) -- (9.25,9.25);
\draw [short] (4.5,9.25) -- (6.5,9.25);
\draw [short] (4.5,9.25) -- (4.25,8.75);
\draw [short] (5,9.25) -- (4.75,8.75);
\draw [short] (5.5,9.25) -- (5.25,8.75);
\draw [short] (6,9.25) -- (5.75,8.75);
\draw [short] (6.5,9.25) -- (6.25,8.75);
\draw [->, >=Stealth] (8.75,9.25) -- (6.75,11.5);
\draw [->, >=Stealth] (8.75,13.5) -- (8.75,11.5);
\node [font=\normalsize] at (4.5,9.5) {P};
\node [font=\normalsize] at (8.75,14) {Q};
\node [font=\normalsize] at (8.75,11.25) {W};
\node [font=\normalsize] at (7.5,10) {R};
\end{circuitikz}


    		\caption{}
    		\label{36}
	\end{figure}

	
	\item
	A uniform beam weighing $1800 N$ is supported at $E$ and $F$ by cable $ABCD$. Determine the tension (in $N$) in segment $AB$ of this cable (correct to 1-decimal place). Assume the cables $ABCD$, $BE$ and $CF$ to be weightless
	
	%fig here
	\begin{figure}[H]
    		\centering
    		\input{figs/figs3.tex}
    		\caption{}
    		\label{36}
	\end{figure}

	
	\item
	Beam $PQRS$ has internal hinges in spans $PQ$ and $RS$ as shown. The beam may be subjected to a moving distributed vertical load of maximum intensity $4 kN/m$ of any length anywhere on the beam. The maximum absolute value of the shear force (in $kN$) that can occur due to this loading just to the right of support $Q$ shall be:
	\begin{figure}[H]
    		\centering
    		\input{figs/figs7.tex}
    		\caption{}
    		\label{36}
	\end{figure}

		\begin{enumerate}
			\item 30
			\item 40
			\item 45
			\item 55
		\end{enumerate}
	\item
	A rectangular concrete beam $250 mm$ wide and $600 mm$ deep is pre-stressed by means of 16 high tensile wires, each of $7 mm$ diameter, located at $200 mm$ from the bottom face of the beam at a given section. If the effective pre-stress in the wires is $700 MPa$, what is the maximum sagging bending moment (in $kNm$) (correct to 1-decimal place) due to live load that this section of the beam can withstand without causing tensile stress at the bottom face of the beam? Neglect the effect of dead load of beam
	\item
	The soil profile below a lake with water level at $elevation = 0m$ and lake bottom at $elevation = -10 m$ is shown in the figure, where $k$ is the permeability coefficient. A piezometer (stand pipe) installed in the sand layer shows a reading of $+10 m$ elevation. Assume that the piezometric head is uniform in the sand layer. The quantity of water (in $m^3/s$) flowing into the lake from the sand layer through the silt layer per unit area of the lake bed is:
	\begin{figure}[H]
    		\centering
    		\input{figs/figs4.tex}
    		\caption{}
    		\label{36}
	\end{figure}
		\begin{enumerate}
			\item $1.5 \times 10^{-6}$
			\item $2.0 \times 10^{-6}$
			\item $1.0 \times 10^{-6}$
			\item $0.5 \times 10^{-6}$
		\end{enumerate}

	\item
	The soil profile above the rock surface for a $25 \degree$ infinite slope is shown in the figure, where $s_u$ is the undrained shear strength and $\gamma_{t}$ is total unit weight. The slip will occur at a depth of
	\begin{figure}[H]
    		\centering
    		\input{figs/figs5.tex}
    		\caption{}
    		\label{36}
	\end{figure}
		\begin{enumerate}
			\item $8.83 m$
			\item $9.79 m$
			\item $7.83 m$
			\item $6.53 m$
		\end{enumerate}
	\item
	Two different soil types (Soil 1 and Soil 2) are used as backfill behind a retaining wall as shown in the figure, where $\gamma_{t}$ is total unit weight, and $c^{\prime}$ and $\psi^{\prime}$ are effective cohesion and effective angle of shearing resistance. The resultant active earth force per unit length (in $kN/m$) acting on the wall is:
	\begin{figure}[H]
    		\centering
    		\input{figs/figs6.tex}
    		\caption{}
    		\label{36}
	\end{figure}
		\begin{enumerate}
			\item 31.7
			\item 35.2
			\item 51.8
			\item 57.0
		\end{enumerate}
	\item
	A $2 km$ long pipe of $0.2 m$ diameter connects two reservoirs. The difference between water levels in the reservoirs is $8 m$. The Darcy-Weisbach friction factor of the pipe is 0.04. Accounting for frictional, entry and exit losses, the velocity in the pipe (in $m/s$) is:
		\begin{enumerate}
			\item 0.63
			\item 0.35
			\item 2.52
			\item 57.0
		\end{enumerate}
		
	
\end{enumerate}
\end{document}
