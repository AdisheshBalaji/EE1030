\let\negmedspace\undefined
\let\negthickspace\undefined
\documentclass[journal]{IEEEtran}
\usepackage[a5paper, margin=10mm, onecolumn]{geometry}
%\usepackage{lmodern} % Ensure lmodern is loaded for pdflatex
\usepackage{tfrupee} % Include tfrupee package

\setlength{\headheight}{1cm} % Set the height of the header box
\setlength{\headsep}{0mm}     % Set the distance between the header box and the top of the text

\usepackage{gvv-book}
\usepackage{gvv}
\usepackage{cite}
\usepackage{amsmath,amssymb,amsfonts,amsthm}
\usepackage{algorithmic}
\usepackage{graphicx}
\usepackage{textcomp}
\usepackage{xcolor}
\usepackage{txfonts}
\usepackage{listings}
\usepackage{enumitem}
\usepackage{mathtools}
\usepackage{gensymb}
\usepackage{comment}
\usepackage[breaklinks=true]{hyperref}
\usepackage{tkz-euclide} 
\usepackage{listings}
% \usepackage{gvv}                                        
\def\inputGnumericTable{}                                 
\usepackage[latin1]{inputenc}                                
\usepackage{color}                                            
\usepackage{array}                                            
\usepackage{longtable}                                       
\usepackage{calc}                                             
\usepackage{multirow}                                         
\usepackage{hhline}                                           
\usepackage{ifthen}                                           
\usepackage{lscape}
\begin{document}
\bibliographystyle{IEEEtran}
\title{Ch.5-Mathematical Induction and Binomial Theorem}
\author{AI24BTECH11016-Jakkula Adishesh Balaji}
{\let\newpage\relax\maketitle}
\renewcommand{\thefigure}{\theenumi}
\renewcommand{\thetable}{\theenumi}
\setlength{\intextsep}{10pt} % Space between text and floats
\numberwithin{equation}{enumi}
\numberwithin{figure}{enumi}
\renewcommand{\thetable}{\theenumi}
Section-A | JEE Advanced/IIT-JEE\\
 I.Integer Value Correct Type
\begin{enumerate}
        \item
            The coefficients of three consecutive terms of $\brak{1+x}^{n+5}$ are in the ratio 5:10:14. Then n=       
                     \hfill(JEE Adv. 2013)
        \item
            Let m be the smallest positive integer such that the coefficient of $x^{2}$ in the expansion of $\brak{1+x}^{2} + \brak{1+x}^{3} + ... + \brak{1+x}^{49} + \brak{1+x}^{50} + \brak{1+mx}^{50}$ is $\brak{3n+1} \comb{51}{3}$ for some $positive integer n. Then the value of n is$ \\
                     \hfill(JEE Adv. 2016)
        \item
            Let \\ $X= \brak{\comb{10}{1}^{2}}+2\brak{\comb{10}{2}^{2}} + 3\brak{\comb{10}{3}^{2}} + ... + 10\brak{\comb{10}{10}^{2}}$, where $\comb{10}{r} , r \in \cbrak{1,2,...,10}$ denote binomial coefficients. Then, the value of $\frac{1}{1430}$X is \rule{10mm}{0.15mm} \\
                    \hfill(JEE Adv. 2018)
        \item
        Suppose \\
           det$\myvec{ \sum\limits_{k=0}^{n}k & \sum\limits_{k=0}^{n} k^{2} \comb{n}{k} \\ \sum\limits_{k=0}^{n}\comb{n}{k} k & \sum\limits_{k=0}^{n}\comb{n}{k} 3^{k}}=0$ \\
holds for some positive integer n. The $\sum\limits_{k=0}^{n} \frac{\comb{n}{k}}{k+1}$ equals \rule{10mm}{0.15mm} \\
                    \hfill(JEE Adv. 2019)
\end{enumerate}
\newpage
\title{Ch.18-Definite Integrals and Applications of Integrals}
\maketitle
\bigskip
Section-B | JEE Main / AIEEE
	\begin{enumerate}
    [start=16]
	      \item 
		The area of the region bounded by the curves $y=\abs{x-2}, x=1, x=3$ and the x-axis is \quad \quad \quad
        \quad \quad \hfill \brak{2004}
		     \begin{enumerate}
		              \item 4
		              \item 2
		              \item 3
		              \item 1
		     \end{enumerate}     
	      \item
           If $I_1=\int_{0}^{1} 2^{x^{2}} \,dx,I_2=\int_{0}^{1} 2^{x^{3}} \,dx,I_3=\int_{1}^{2} 2^{x^{2}} \,dx, I_4=\int_{1}^{2} 2^{x^{3}} \, dx
          $ then 
          \hfill \brak{2005} 
          \begin{enumerate}
                   \item $I_2>I_1$
                   \item $I_1>I_2$
                   \item $I_3=I_4$
                   \item $I_3>I_4$
          \end{enumerate}
	      \item
		      The area enclosed between the curve $y=\log_e\brak{x+e}$ and the coordinate  axes is 
		     \hfill \brak{2005} 
		     \begin{enumerate}
		              \item 1
		              \item 2
		              \item 3
		              \item 4
		     \end{enumerate}
	      \item 
		      The parabolas $y^{2}=4x$ and $x^{2}=4y$ divide the square region bounded by the lines $x=4, y=4$ and the coordinate axes. If $S_1, S_2, S_3$ are respectively the areas of these parts numbered from top to bottom then $S_1:S_2:S_3$ is
		     \hfill \brak{2005} 
		     \begin{enumerate}
		             \item 1:2:1
		             \item 1:2:3
		             \item 2:1:2
		             \item 1:1:1
		     \end{enumerate}  
	     \item 
		      Let $f\brak{x}$ be a non-negative continuous function such that the area bounded by the curve $y=f\brak{x}$, x-axis and the oordinates $x=\frac{\pi}{4}$ and $x=\beta>\frac{\pi}{4}$ is $\brak{\brak{\beta \sin\brak{\beta}+\frac{\pi}{4}\cos\brak{\beta}}}$. Then $f\brak{\frac{\pi}{2}}$ is
		     \hfill \brak{2005}
		     \begin{enumerate}
		            \item $\frac{\pi}{4}+\sqrt{2}-1$
		            \item $\frac{\pi}{2}-\sqrt{2}+1$
		            \item $1-\frac{\pi}{4}-\sqrt{2}$
		            \item $1-\frac{\pi}{4}+\sqrt{2}$
		     \end{enumerate}
	     \item 
		      The value of $\int_{-\pi}^{\pi} \frac{\cos^{2}\brak{x}}{1+a^{x}} \quad dx$, $a>0$, is
		     \hfill \brak{2005}
		     \begin{enumerate}
		           \item $\pi$
		           \item $\frac{\pi}{2}$
		           \item $\frac{\pi}{a}$
		           \item $2\pi$
		     \end{enumerate}
	     \item 
		     The value of the integral $\int_{3}^{6} \frac{\sqrt{x}}{\sqrt{9-x}+\sqrt{x}} \quad dx$ is \quad \quad \quad
       \quad \quad
		     \hfill \brak{2005} 
		    \begin{enumerate}
		     
		          \item $\frac{1}{2}$
		          \item $\frac{3}{2}$
		          \item 2
		          \item 1
		    \end{enumerate}
	     \item 
		     $\int_{0}^{\pi} xf\brak{\sin x} \quad dx$ is equal to 
		    \hfill \brak{2006}
		    \begin{enumerate}
		    
		         \item $\pi \int_{0}^{\pi} f\brak{\cos x} \quad dx$
		         \item $\pi \int_{0}^{\pi} f\brak{\sin x} \quad dx$
		         \item $\frac{\pi}{2} \int_{0}^{\frac{\pi}{2}} f\brak{\sin x} \quad dx$
		         \item $\pi \int_{0}^{\frac{\pi}{2}} f\brak{\cos x} \quad dx$
		    \end{enumerate}
	     \item
		    $\int_{\frac{-3\pi}{2}}^{\frac{-\pi}{2}} \sbrak{\brak{x+\pi}^{3}+\cos^{2}\brak{x+3\pi}} \quad dx$ is equal to
		    \hfill \brak{2006} 
		   \begin{enumerate}
		        \item $\frac{\pi^{4}}{32}$
		        \item $\frac{\pi^{4}}{32}+\frac{\pi}{2}$
		        \item $\frac{\pi}{2}$
		        \item $\frac{\pi}{4}-1$
		   \end{enumerate}
	    \item
		    The value of $\int_{1}^{a} [x]f^{'}(x) \quad dx$, $a>1$ where [x] denotes the greatest integer not exceeding x is
		   \hfill \brak{2006}
		   \begin{enumerate}
		       \item $af\brak{a}-\cbrak{f\brak{1}+f\brak{2}+...f\brak{\sbrak{a}}}$
		       \item $\sbrak{a}f\brak{a}-\cbrak{f\brak{1}+f\brak{2}+...f\brak{\sbrak{a}}}$
		       \item $\sbrak{a}f\brak{\sbrak{a}}-\cbrak{f\brak{1}+f\brak{2}+...f\brak{a}}$
		       \item $af\brak{\sbrak{a}}-\cbrak{f\brak{1}+f\brak{2}+...f\brak{a}}$
		   \end{enumerate}
	   \item 
		    Let $F\brak{x}=f\brak{x}+f\brak{\frac{1}{x}}$, where $f\brak{x}=\int_{1}^{x} \frac{logt}{1+t} \quad dt$, Then $F\brak{e}$ equals
		    \hfill \brak{2006} 
		   \begin{enumerate}
		      \item 1
		      \item 2
		      \item $\frac{1}{2}$
		      \item 0
		   \end{enumerate}  
	   \item 
    		    The solution for x of the equation $\int_{\sqrt2}^{x} \frac{1}{t\sqrt{t^{2}-1}} \quad dt$ is
    		   \hfill \brak{2007}
		   \begin{enumerate}
		      \item $\frac{\sqrt{3}}{2}$
		      \item $2\sqrt{2}$
		      \item 2
		      \item none
		   \end{enumerate} 
	   \item 
		  The area enclosed between the curves $y^{2}=x$ and $y=\abs{x}$ is
		 \hfill \brak{2007}
		 \begin{enumerate}
		     \item $\frac{1}{6}$
		     \item $\frac{1}{3}$
		     \item $\frac{2}{3}$
		     \item 1
		 \end{enumerate}
	   \item 
		 Let $I=\int_{0}^{1} \frac{\sin x}{\sqrt{x}} \,dx$ and $J=\int_{0}^{1} \frac{\cos x}{\sqrt{x}} \quad dx$. Then which one of the following is true?
		\hfill \brak{2007} 
		\begin{enumerate}
		    \item $I>\frac{2}{3}$ and $J>2$
		    \item $I<\frac{2}{3}$ and $J<2$
		    \item $I<\frac{2}{3}$ and $J>2$
		    \item $I>\frac{2}{3}$ and $J<2$
		\end{enumerate}
	   \item 
		 The area of the plane region bounded by the curves $x+2y^{2}=0$ and $x+3y^{2}=1$ is equal to  
		\hfill \brak{2008}
		\begin{enumerate}
		   \item $\frac{5}{3}$
		   \item $\frac{1}{3}$
		   \item $\frac{2}{3}$
		   \item $\frac{4}{3}$
		\end{enumerate}
	\end{enumerate}
	\end{document}
