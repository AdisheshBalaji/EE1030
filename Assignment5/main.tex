\documentclass[journal]{IEEEtran}
\usepackage[a5paper, margin=10mm, onecolumn]{geometry}
\usepackage{cite}
\usepackage{amsmath,amssymb,amsfonts,amsthm}
\usepackage{algorithmic}
\usepackage{graphicx}
\usepackage{textcomp}
\usepackage{xcolor}
\usepackage{txfonts}
\usepackage{enumitem}
\usepackage{mathtools}
\usepackage{gensymb}
\usepackage{comment}
\usepackage[breaklinks=true]{hyperref}
\usepackage{tkz-euclide}
\usepackage[latin1]{inputenc}                                
\usepackage{color}                                            
\usepackage{array}                                            
\usepackage{longtable}                                       
\usepackage{calc}                                             
\usepackage{multirow}                                         
\usepackage{hhline}                                           
\usepackage{ifthen}                                           
\usepackage{lscape}

\setlength{\headheight}{1cm} % Set the height of the header box
\setlength{\headsep}{0mm}     % Set the distance between the header box and the top of the text

\begin{document}

\bibliographystyle{IEEEtran}
\setlength{\intextsep}{10pt} % Space between text and floats
\numberwithin{equation}{section} % Use section numbering for equations
\numberwithin{figure}{section} % Use section numbering for figures
\numberwithin{table}{section} % Use section numbering for tables

\title{9-9.2-13}
\author{AI24BTECH11016 - Jakkula Adishesh Balaji}
\maketitle
\section*{\textbf{Intersection Of Conics(Chords)}}
\parindent 0pt
\textbf{Question:} \\
\textbf{9.2.13} Find the area of the region bounded by the ellipse $\frac{x^{2}}{16} + \frac{y^{2}}{9} = 1$\\
\begin{table}[h!]    	
    \centering
    % Assuming the table.tex file exists
    \begin{tabular}{|c|c|}
    \hline
    P. Maximum-normal-stress criterion & \input{figs/figs.tex} \\ 
    \hline
    Q. Maximum-distortion-energy criterion & \input{figs/figs1.tex} \\
    \hline
    R. Maximum-shear-stress criterion & 
\begin{circuitikz}[scale = 0.5]
\tikzstyle{every node}=[font=\normalsize]
\draw [short] (5.25,12.25) .. controls (9.25,13.5) and (10.75,6) .. (5.25,6.75);
\draw [short] (8.25,9.25) -- (9.25,9.25);
\draw [short] (4.5,9.25) -- (6.5,9.25);
\draw [short] (4.5,9.25) -- (4.25,8.75);
\draw [short] (5,9.25) -- (4.75,8.75);
\draw [short] (5.5,9.25) -- (5.25,8.75);
\draw [short] (6,9.25) -- (5.75,8.75);
\draw [short] (6.5,9.25) -- (6.25,8.75);
\draw [->, >=Stealth] (8.75,9.25) -- (6.75,11.5);
\draw [->, >=Stealth] (8.75,13.5) -- (8.75,11.5);
\node [font=\normalsize] at (4.5,9.5) {P};
\node [font=\normalsize] at (8.75,14) {Q};
\node [font=\normalsize] at (8.75,11.25) {W};
\node [font=\normalsize] at (7.5,10) {R};
\end{circuitikz}

 \\
    \hline
\end{tabular}

 
    \caption{Parameters Used}
    \label{tab:1-1.9-6}
\end{table}
\textbf{Solution:}
The area under the curve is given by \\
\begin{align}
A &= 4 \int_0^4 b \sqrt{1 - \frac{x^2}{a^2}} \, dx \\
\implies &= 4 \int_0^4 3 \sqrt{1 - \frac{x^2}{16}} \, dx \\
A &= 12\pi
\end{align}
\end{document}


