%iffalse
\documentclass[journal]{IEEEtran}
\usepackage[a5paper, margin=10mm, onecolumn]{geometry}
%\usepackage{lmodern} % Ensure lmodern is loaded for pdflatex
\usepackage{tfrupee} % Include tfrupee package

\setlength{\headheight}{1cm} % Set the height of the header box
\setlength{\headsep}{0mm}     % Set the distance between the header box and the top of the text

\usepackage{gvv-book}
\usepackage{gvv}
\usepackage{cite}
\usepackage{amsmath,amssymb,amsfonts,amsthm}
\usepackage{algorithmic}
\usepackage{graphicx}
\usepackage{textcomp}
\usepackage{xcolor}
\usepackage{txfonts}
\usepackage{listings}
\usepackage{enumitem}
\usepackage{mathtools}
\usepackage{gensymb}
\usepackage{comment}
\usepackage[breaklinks=true]{hyperref}
\usepackage{tkz-euclide} 
\usepackage{listings}
% \usepackage{gvv}                                        
\def\inputGnumericTable{}                                 
\usepackage[latin1]{inputenc}                                
\usepackage{color}                                            
\usepackage{array}                                            
\usepackage{longtable}                                       
\usepackage{calc}                                             
\usepackage{multirow}                                         
\usepackage{hhline}                                           
\usepackage{ifthen}                                           
\usepackage{lscape}
\begin{document}
\bibliographystyle{IEEEtran}
\title{2020-Sept Session-09-04-2020 shift 2}
\author{AI24BTECH11016-Jakkula Adishesh Balaji}
{\let\newpage\relax\maketitle}
\renewcommand{\thefigure}{\theenumi}
\renewcommand{\thetable}{\theenumi}
\setlength{\intextsep}{10pt} % Space between text and floats
\numberwithin{equation}{enumi}
\numberwithin{figure}{enumi}
\renewcommand{\thetable}{\theenumi}
\section{1-15(Math)}
\begin{enumerate}
	\item
	Suppose the vectors $x_1$, $x_2$ and $x_3$ are the solutions of the system of linear equations, $\vec{A}\vec{x} = b$ when the vector $b$ on the right side is equal to $b_1$, $b_2$ and $b_3$ respectively. If $x_1 = \myvec{1 \\ 1 \\ 1}$, $x_2 = \myvec{0 \\ 2 \\ 1}$, $x_3 = \myvec{0 \\ 0 \\ 1}$, $b_1 = \myvec{1 \\ 0 \\ 0}$, $b_2 = \myvec{0 \\ 2 \\ 0}$, $b_3 = \myvec{0 \\ 2 \\2}$, then the determinant of $A$ is equal to 
		\begin{enumerate}
			\item 2
			\item $\frac{1}{2}$
			\item $\frac{3}{2}$
			\item 4
		\end{enumerate}
	\item
	 If $a$ and $b$ are real numbers such that $\brak{2+\alpha}^{4} = a+b\alpha$, where $\alpha = \frac{-1+i\sqrt{3}}{2}$ then $a+b$ is equal to:
	 	\begin{enumerate}
	 		\item 33
	 		\item 57
	 		\item 9
	 		\item 24
	 	\end{enumerate}
	\item
	The distance of the point $\myvec{1 \\ -2 \\ 3}$ from the plane $x-y+z = 5$ measured parallel to the line $\frac{x}{2} = \frac{y}{3} = \frac{z}{-6}$ is:
		\begin{enumerate}
			\item $\frac{1}{7}$
			\item 7
			\item $\frac{7}{5}$
			\item 1
		\end{enumerate}
	\item
	Let $f: (0, \infty) \to (0, \infty)
$ be a differentiable function such that $f\brak{1} = e$ and \\
	$\lim_{t \to x} \frac{t^{2}f^{2}\brak{x}-x^{2}f^{2}\brak{t}}{t-x} = 0$ \\
	If $f\brak{x} = 1$, then $x$ is equal to:
		\begin{enumerate}
			\item e
			\item 2e
			\item $\frac{1}{e}$
			\item $\frac{2}{e}$
		\end{enumerate}
	\item
	Contrapositive of the statement : \\
	If a function $f$ is differentiable at $a$, then it is also continuous at $a$, is:
		\begin{enumerate}
			\item If a function $f$ is not continuous at $a$, then it is not differentiable at $a$.
			\item If a function $f$ is continuous at $a$, then it is differentiable at $a$.
			\item If a function $f$ is continuous at $a$, then it is not differentiable at $a$.
			\item If a function $f$ is continuous at $a$, then it is not differentiable at $a$.
		\end{enumerate}
	\item
	The minimum value of $2^{\sin\brak{x}}+2^{\cos\brak{x}}$ is:
		\begin{enumerate}
			\item $2^{1-\sqrt{2}}$
			\item $2^{1 - \frac{1}{\sqrt{2}}}$
			\item $2^{-1+\sqrt{2}}$
			\item $2^{-1 + \frac{1}{\sqrt{2}}}$
		\end{enumerate}
	\item
	If the perpendicular bisector of the line segment joining the points $P\brak{1 ,4}$ and $Q\brak{k, 3}$ has y-intercept equal to -4, then a value of $k$ is:
		\begin{enumerate}
			\item -2
			\item $\sqrt{15}$
			\item $\sqrt{14}$
			\item -4
		\end{enumerate}
	\item
	The area \brak{in sq. units} of the largest rectangle $ABCD$ whose vertices $A$ and $B$ lie on the $x-axis$ and vertices $C$ and $D$ lie on the parabola, $y = x^{2}-1$ below 			the $x-axis$, is:
		\begin{enumerate}
			\item $\frac{2}{3\sqrt{3}}$
			\item $\frac{4}{3}$
			\item $\frac{1}{3\sqrt{3}}$
			\item $\frac{4}{3\sqrt{3}}$
		\end{enumerate}
	\item
	The integral \\
	$\int_{\frac{\pi}{6}}^{\frac{\pi}{3}} \tan^{3}{x}\sin^{2}{3x}\brak{2\sec^{2}{x}\sin^{2}{3x} + 3\tan{x}\sin{6x}} dx$ is equal to 
		\begin{enumerate}
			\item $\frac{9}{2}$
			\item $\frac{-1}{18}$
			\item $\frac{-1}{9}$
			\item $\frac{7}{18}$
		\end{enumerate}
	\item
	If the system of equations
	\begin{align}
	x + y + z &= 2 \\
	2x + 4y - z &= 6 \\
	3x + 2y + \lambda z &=\mu
	\end{align}
	has infinitely many solutions, then
		\begin{enumerate}
			\item $\lambda - 2\mu = -5$
			\item $2\lambda + \mu = 14$
			\item $\lambda + 2\mu = 14$
			\item $2\lambda - \mu = 5$
		\end{enumerate}
	\item 
	In a game two players A and B take turns in throwing a pair of fair dice starting with player A and total of scores on the two dice, in each throw is noted. A wins the game if he throws a total of 6 before B throws a total of 7 and B wins the game if he throws a total of 7 before A throws a total of six The game stops as soon as either of the players wins. The probability of A winning the game is :
		\begin{enumerate}
			\item $\frac{5}{31}$
			\item $\frac{31}{61}$
			\item $\frac{30}{61}$
			\item $\frac{5}{6}$
		\end{enumerate}
	\item
	If for some positive integer $n$, the coefficients of three consecutive terms in the binomial expansion of $\brak{1+x}^{n+5}$ are in the ratio $5:10:14$, then the largest coefficient in this expansion is :
		\begin{enumerate}
			\item 792
			\item 252
			\item 462
			\item 330
		\end{enumerate}
	\item
	The function \\
	$f\brak{x} =
	\begin{cases}
    		\frac{\pi}{4} + \tan^{-1}{x} & \text{if } \abs{x} \leq 1 \\
    		\frac{1}{2}(\abs{x}-1) & \text{if } \abs{x} > 1
	\end{cases}
	$
		\begin{enumerate}
			\item  both continuous and differentiable on $R-{-1}$
			\item continuous on $R-{-1}$ and differentiable on $R-{-1,1}$
			\item continuous on $R-{1}$ and differentiable on $R-{-1,1}$
			\item both continuous and differentiable on $R-{1}$
		\end{enumerate}
	\item 
	The solution of the differential equation $\frac{dy}{dx}-\frac{y+3x}{\log_e{y+3x}} + 3 = 0$ is: where $c$ is a constant of integration
		\begin{enumerate}
			\item $x - \log_{e}{y+3x} = c$
			\item $x - \frac{1}{2}\brak{{\log_{e}{y+3x}}^2} = c$
			\item $x - 2\log_{e}{y+3x} = c$
			\item $y + 3x - \frac{1}{2}\log_{e}{x}^{2} = c$
		\end{enumerate}
	\item 
	Let $\lambda \neq 0$ be in $R$. If $\alpha$ and $\beta$ are the roots of the equation, $x^{2}-x+2\lambda = 0$ and $\alpha$ and $gamma$ are the roots of the equation, $3x^{2}-10x+27\lambda = 0$, then $\frac{\beta\gamma}{\lambda}$ is equal to:
		\begin{enumerate}
			\item 27
			\item 9
			\item 18
			\item 36
		\end{enumerate}
\end{enumerate}
\end{document}
		

	
	
		
